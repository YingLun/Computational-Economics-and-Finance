\lstset{language=Matlab,%
    %basicstyle=\color{red},
    basicstyle=\footnotesize,
    breaklines=true,%
    morekeywords={matlab2tikz},
    keywordstyle=\color{blue},%
    morekeywords=[2]{1}, keywordstyle=[2]{\color{black}},
    identifierstyle=\color{black},%
    stringstyle=\color{mylilas},
    commentstyle=\color{mygreen},%
    showstringspaces=false,%without this there will be a symbol in the places where there is a space
    numbers=left,%
    numberstyle={\tiny \color{black}},% size of the numbers
    numbersep=9pt, % this defines how far the numbers are from the text
    emph=[1]{for,end,break},emphstyle=[1]\color{red}, %some words to emphasise
    %emph=[2]{word1,word2}, emphstyle=[2]{style},    
}

\section*{Main Code}
\subsection*{Question 1}
\lstinputlisting[language=Matlab]{Code/Q1/Problem_3.m}

\subsection*{Question 2}
\lstinputlisting[language=Matlab]{Code/Q2/Problem_2.m}

\subsection*{Question 3}
\lstinputlisting[language=Matlab]{Code/Q3/Problem_3.m}

\subsection*{Question 4}
\lstinputlisting[language=Matlab]{Code/Q4/Q4.m}

\subsection*{Question 5}
\lstinputlisting[language=Matlab]{Code/Q5/Q5.m}

\section*{Functions}
\subsection*{Question 2}
\lstinputlisting[language=Matlab]{Code/Q2/bisection.m}
\subsection*{Question 4}
\lstinputlisting[language=Matlab]{Code/Q4/Broyden_Method.m}
\lstinputlisting[language=Matlab]{Code/Q4/Inverse_Broyden_Method.m}
\lstinputlisting[language=Matlab]{Code/Q4/Jacobian.m}
\lstinputlisting[language=Matlab]{Code/Q4/Newton_Method.m}
\lstinputlisting[language=Matlab]{Code/Q4/steady_state_fixed_pt.m}
\lstinputlisting[language=Matlab]{Code/Q4/steady_state.m}